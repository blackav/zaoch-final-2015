\begin{problem}{Ремонт асфальта}{stdin}{stdout}{2 секунды}{256 мегабайт}

Улица М. города Д. печально известна среди местных жителей качеством дорожного покрытия. В этом тяжело винить ремонтные службы: пожалуй, они следят за этой улицей даже слишком тщательно. Проблема в том, что каждый без исключения ремонт улицы выглядит следующим образом: бригада рабочих выбирает некоторый небольшой участок улицы и меняет асфальт только на нём, причём тип асфальта на этом участке в результате может отличаться от типов асфальта на других участках, что, разумеется, усложняет проезд по улице.

Вы, как коренной житель города Д. и программист по призванию, решили использовать свои профессиональные навыки на благо общества и облегчить жизнь своим соседям по улице М. А именно, вы решили создать сайт, содержащий актуальную информацию о \emph{непроходимости} улицы. Прежде всего, вы заметили, что улица разбита на $N$ идущих друг за другом участков единичной длины. По странному совпадению, бригада рабочих всегда выбирает для ремонта ровно один из таких участков и целиком меняет тип асфальта на нём. Затем вы пронумеровали эти участки от 1 до $N$ и собрали информацию о типе асфальта на каждом из участков~--- числа $t_1$, $t_2$, \ldots, $t_N$ ($t_i$~--- номер типа асфальта на $i$-м участке согласно Государственному реестру дорожных покрытий). Наконец, вы определили \emph{непроходимость} улицы как минимальное количество непрерывных непересекающихся отрезков c одинаковым типом асфальта, на которые она разбивается. Например, непроходимость улицы \t{110111} равна 3, потому что она состоит из трёх участков \t{11}, \t{0} и \t{111}, а идеальная улица \t{2222} имеет непроходимость, равную 1.

Казалось бы, достаточно вычислить и разместить на сайте текущую величину непроходимости улицы, и жители будут довольны? К сожалению, асфальт меняют довольно часто, и вам не хочется каждый раз выходить на улицу и заново собирать данные. Поэтому вы дали возможность жителям сообщать на вашем сайте об обновлении дорожного покрытия. Дело осталось за малым: после каждого такого сообщения научиться обновлять актуальную величину непроходимости улицы.

\InputFile
Первая строка входного файла содержит единственное натуральное число $N$~--- количество участков дороги  ($1 \le N \le 100\,000$). Следующая строка содержит $N$ целых чисел $t_1$, $t_2$, \ldots, $t_N$~--- исходные типы асфальта участков дороги ($|t_i| \le 10^9$).

Третья строка содержит единственное натуральное число $Q$~--- количество сообщений от жителей об обновлении дорожного покрытия ($1 \le Q \le 100\,000$). Каждая из следующих $Q$ строк содержит очередное сообщение.

$i$-е сообщение представляет собой пару целых чисел $p_i$, $c_i$~--- номер ремонтируемого участка дороги и новый номер типа асфальта на этом участке ($1 \le p_i \le N$, $|c_i| \le 10^9$). Участки дороги нумеруются от 1 до $N$ в порядке задания их исходного типа асфальта во второй строке входного файла.

\OutputFile
Выведите $N$ строк. $i$-я строка ($1 \le i \le N$) должна содержать единственное целое число~--- величину непроходимости улицы после первых $i$ обновлений дорожного покрытия.

\Examples

\begin{example}
\exmp{5
2 2 2 2 2
5
1 2
2 3
4 3
3 1
3 3
}{1
3
5
5
3
}%
\exmp{7
1 1 2 3 2 2 1
3
2 2
4 2
6 9
}{5
3
4
}%
\end{example}

\Note
Рассмотрим подробнее второй тестовый пример. Изначально улица \t{1123221} состоит из 5 отрезков с одинаковым типом асфальта: \t{11}, \t{2}, \t{3}, \t{22}, \t{1} и, соответственно, имеет непроходимость, равную 5 (её не нужно выводить в выходной файл).

После первого ремонта улица станет выглядеть как \t{1223221} и всё ещё будет состоять из 5 участков, но других: \t{1}, \t{22}, \t{3}, \t{22}, \t{1}. Поэтому её непроходимость равна 5, и первое число в выходном файле равно 5.

После второго ремонта улица будет состоять из 3 участков: \t{1}, \t{22222}, \t{1}, так что второе число в выходном файле~--- 3.

После третьего ремонта получим 4 участка: \t{1}, \t{2222}, \t{9}, \t{1}, соответственно, третье и последнее число в выходном файле~--- 4.

\Scoring

Тесты к этой задаче состоят из трёх групп. Баллы за каждую группу тестов ставятся только при прохождении всех тестов группы и \textbf{всех тестов предыдущих групп.}
\medskip

\begingroup
\renewcommand{\arraystretch}{1.5}

\begin{tabular}{|c|c|c|c|c|}
\hline
& & & & \\
\raisebox{2.25ex}[0cm][0cm]{Группа} & 
\raisebox{2.25ex}[0cm][0cm]{Тесты} &
\raisebox{2.25ex}[0cm][0cm]{Баллы} &
\raisebox{2.25ex}[0cm][0cm]{Дополнительные ограничения} & \raisebox{2.25ex}[0cm][0cm]{Комментарий} \\
\hline
0 & 1--2 & 0                     & --                   & Тесты из условия \\
\hline
1 & 3--20 & 50 & $N, Q \leq 1\,000$ & \\
\hline
2 & 21--30 & 50 & -- & Дополнительных ограничений нет \\
\hline
\end{tabular}
\endgroup


\end{problem}
