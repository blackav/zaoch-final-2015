\begin{problem}{Пожар в НИИЧАВО}{stdin}{stdout}{15 секунд}{512 мегабайт}

В научно-исследовательском институте чародейства и волшебства пожар! Во время опыта Корнеева В. П. по превращению всей морской и океанской воды планеты в живую воду произошло короткое замыкание, и теперь его кабинет объят пламенем. Задача первостепенной важности --- спасти из огня ценные лабораторные приборы, в особенности единственный в своём роде диван-транслятор М-поля. Ваша задача --- перенести диван-транслятор из кабинета Корнеева в запасную лабораторию изучения М-поля.

НИИЧАВО состоит из $N$ кабинетов, соединённых $M$ коридорами. Кабинеты пронумерованы целыми числами от $1$ до $N$, при этом кабинет Корнеева имеет номер $A$, а лаборатория изучения М-поля расположена в кабинете номер $B$. Благодаря специальному искажению пространства внутри института, все коридоры имеют одинаковую длину, которую можно пройти за $1$ минуту, если двигаться быстрым шагом.

Ситуация усугубляется тем, что диван-транслятор --- прибор, очень чувствительный к резким перепадам температуры. Внутри каждого коридора НИИЧАВО поддерживается свой температурный режим. Если абсолютная величина разности температур в двух \textbf{последовательных} коридорах на пути из кабинета Корнеева в лабораторию окажется больше $D$ градусов, то диван-транслятор перейдёт в нестабильное состояние, что может привести к катастрофическим последствиям. Обратите внимание, что на своём пути вы не заходите в сами кабинеты, а только переходите из коридора в коридор, поэтому климат внутри кабинетов не влияет на диван-транслятор. В силу причин магического характера, войдя в коридор, вы обязаны дойти до его конца, иными словами, останавливаться или разворачиваться посреди коридора запрещено. По каждому коридору можно перемещаться в обоих направлениях.

Определите, за какое минимальное время можно добраться из кабинета Корнеева до запасной лаборатории, не допуская резкого перепада температур. В рамках данной задачи вам предлагается ответить на поставленный вопрос для нескольких наборов значений $A$, $B$ и $D$.

\InputFile
В первой строке входных данных следуют два целых числа $N$ и $M$ ($2 \leq N \leq 100\,000$, $1 \leq M \leq 1\,000\,000$), обозначающие количество кабинетов и количество коридоров в НИИЧАВО.

В последующих $M$ строках находятся описания коридоров. Каждая строка содержит по три целых числа $u_i$, $v_i$, $t_i$~--- номера двух кабинетов, соединённых $i$-ым коридором, и значение температуры в этом коридоре, выраженное в градусах ($1 \leq u_i, v_i \leq N$, $-10^9 \leq t_i \leq 10^9$). Как вы уже могли понять, НИИЧАВО --- весьма необычное заведение, поэтому между двумя кабинетами может пролегать несколько коридоров, возможно с разными температурами, а некоторые коридоры могут соединять кабинет с самим собой. Гарантируется, что коридоры перечислены во входном файле в порядке \textbf{неубывания} $t_i$.

В следующей строке находится целое число $Q$ ($1 \leq Q \leq 10$) --- количество наборов данных, которые вам требуется обработать.

В каждой из последующих $Q$ строк находится по три целых числа $A_i$, $B_i$, $D_i$, обозначающих номер кабинета Корнеева, номер кабинета, в котором расположена лаборатория, и максимальный допустимый перепад температур для дивана-транслятора в градусах ($1 \leq A_i, B_i \leq N$, $A_i \neq B_i$, $0 \leq D_i \leq 2 \cdot 10^9$).

\OutputFile
Для каждого набора данных выведите в отдельной строке минимальное количество минут, которое требуется потратить, чтобы добраться из кабинета Корнеева до лаборатории, либо выведите $-1$, если сделать это, используя допустимый для дивана-транслятора маршрут, невозможно.

\Examples

\begin{example}
\exmp{6 8
1 2 4
2 3 6
3 2 7
6 1 11
2 5 12
1 3 14
3 4 16
5 6 17
3
1 5 5
4 2 4
4 2 6
}{4
-1
5
}%
\end{example}

\Note
Пояснение к первому тесту из условия.

В первом наборе $A = 1$, $B = 5$, $D = 5$. В качестве возможного маршрута может выступить следующая последовательность переходов по коридорам: 
$1 \xrightarrow{t = 4^\circ} 
 2 \xrightarrow{t = 6^\circ} 
 3 \xrightarrow{t = 7^\circ} 
 2 \xrightarrow{t = 12^\circ} 
 5$.

Во втором наборе $A = 4$, $B = 2$, $D = 4$. Способа добраться из кабинета $4$ в кабинет $2$, ни разу не допустив перепад температуры больше, чем в $4$ градуса, не существует.

В третьем наборе $A = 4$, $B = 2$, $D = 7$. Стартовый и конечный кабинет те же, что и в предыдущем наборе, но допустимый перепад температуры больше, благодаря чему подходит следующий маршрут: 
$4 \xrightarrow{t = 16^\circ}
 3 \xrightarrow{t = 14^\circ}
 1 \xrightarrow{t = 11^\circ}
 6 \xrightarrow{t = 17^\circ}
 5 \xrightarrow{t = 12^\circ}
 2$.

\Scoring

Тесты к этой задаче состоят из дохерищи групп. Баллы за~каждую группу ставятся только при прохождении всех тестов группы и \textbf{всех тестов предыдущих} групп. \textbf{Offline-проверка} означает, что результаты тестирования вашего решения на данной группе станут доступны только после окончания соревнования.


\medskip

\begingroup
\renewcommand{\arraystretch}{1.5}

\begin{tabular}{|c|c|c|c|c|c|c|}
\hline
& & & \multicolumn{3}{c|}{Дополнительные ограничения} & \\
\cline{4-6}
\raisebox{2.25ex}[0cm][0cm]{Группа} & 
\raisebox{2.25ex}[0cm][0cm]{Тесты} &
\raisebox{2.25ex}[0cm][0cm]{Баллы} &
$N$ & $M$ & $Q$ & \raisebox{2.25ex}[0cm][0cm]{Комментарий} \\
\hline
0 & ?--?   & 0  & --            & --            & --              & Тесты из условия. \\
\hline
1 & ??--??  & 30 & --    & --    & --  &                  \\
\hline
2 & ??--?? & 30 & --    & --    & --              &                  \\
\hline
3 & ??--?? & hz & --    & --            & --              & \textbf{Offline-проверка.}                 \\
\hline
4 & ??--?? & hz & --            & --            & --              & \textbf{Offline-проверка.} \\
\hline
\end{tabular}

\endgroup

\end{problem}
